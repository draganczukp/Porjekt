\documentclass[10pt,a4paper]{article}
%\documentclass[12pt,a4paper]{article}

\usepackage[utf8]{inputenc}
\usepackage{hyperref}
\usepackage{polski}
\usepackage{graphicx}
\usepackage{listings}
\usepackage{algorithm}
\usepackage{algorithmic}

\usepackage{hyperref}
%\usepackage{antpolt}
\usepackage{amssymb}
\usepackage{multicol}
\usepackage{fancyvrb}

\usepackage{color}

%\setlength{\topskip}{0mm} \setlength{\footskip}{0mm} \setlength{\topmargin}{0mm} \setlength{\marginparwidth}{0mm}
%\setlength{\headsep}{2mm} \setlength{\headheight}{0mm} \setlength{\textheight}{250mm}
%\setlength{\textwidth}{160mm} \setlength{\oddsidemargin}{0mm} \setlength{\evensidemargin}{0mm}

\setlength{\topskip}{0mm} \setlength{\topmargin}{0mm}
\setlength{\oddsidemargin}{0mm} \setlength{\evensidemargin}{0mm}
\setlength{\marginparwidth}{0mm} \setlength{\headsep}{0mm}
\setlength{\headheight}{0mm} \setlength{\textheight}{240mm}
\setlength{\textwidth}{170mm}


\floatname{algorithm}{Algorytm}

% \k{a} \'c \k{e} \l{} \'n \'o \'s
% \'z \.z \k{A} \'C \k{E} \L{} \'N
% \'O \'S \'Z \.Z 

\lstset{literate=
  {á}{{\'a}}1 {é}{{\'e}}1 {í}{{\'i}}1 {ó}{{\'o}}1 {ú}{{\'u}}1
  {Á}{{\'A}}1 {É}{{\'E}}1 {Í}{{\'I}}1 {Ó}{{\'O}}1 {Ú}{{\'U}}1
  {à}{{\`a}}1 {è}{{\`e}}1 {ì}{{\`i}}1 {ò}{{\`o}}1 {ù}{{\`u}}1
  {À}{{\`A}}1 {È}{{\'E}}1 {Ì}{{\`I}}1 {Ò}{{\`O}}1 {Ù}{{\`U}}1
  {ä}{{\"a}}1 {ë}{{\"e}}1 {ï}{{\"i}}1 {ö}{{\"o}}1 {ü}{{\"u}}1
  {Ä}{{\"A}}1 {Ë}{{\"E}}1 {Ï}{{\"I}}1 {Ö}{{\"O}}1 {Ü}{{\"U}}1
  {â}{{\^a}}1 {ê}{{\^e}}1 {î}{{\^i}}1 {ô}{{\^o}}1 {û}{{\^u}}1
  {Â}{{\^A}}1 {Ê}{{\^E}}1 {Î}{{\^I}}1 {Ô}{{\^O}}1 {Û}{{\^U}}1
  {Ã}{{\~A}}1 {ã}{{\~a}}1 {Õ}{{\~O}}1 {õ}{{\~o}}1
  {œ}{{\oe}}1 {Œ}{{\OE}}1 {æ}{{\ae}}1 {Æ}{{\AE}}1 {ß}{{\ss}}1
  {ű}{{\H{u}}}1 {Ű}{{\H{U}}}1 {ő}{{\H{o}}}1 {Ő}{{\H{O}}}1
  {ç}{{\c c}}1 {Ç}{{\c C}}1 {ø}{{\o}}1 {å}{{\r a}}1 {Å}{{\r A}}1
  {€}{{\euro}}1 {£}{{\pounds}}1 {«}{{\guillemotleft}}1
  {»}{{\guillemotright}}1 {ñ}{{\~n}}1 {Ñ}{{\~N}}1 {¿}{{?`}}1
  {ł}{\l{}}1
}

\begin{document}
\pagestyle{empty}

%
%  strona tytułowa
%

\begin{center}
\textsc{\Huge{Uniwersytet Zielonogórski}}\\
\LARGE{Wydział Informatyki, Elektrotechniki i~Automatyki}\\
\large{Instytut Sterowania i Systemów Informatycznych}\\
\vspace{0.5cm}
\Large{Programowanie Gier 3D -- Projekt}\\
Prowadzący: Dr inż. Marek Sawerwain \\ \vspace{1cm}
\LARGE{Sprawozdanie z projektu}\\

\vspace{0.5cm} 
\Large{Wykonał: Przemysław Dragańczuk, Grupa dziekańska: 33-INF-SSI-SA} \\

\Large{Data oddania projektu: 23.01.2019}
\vspace{1cm}

\begin{flushleft}
	Ocena: ..........................................
\end{flushleft}

\vspace{1cm}
\end{center}



%
% spis treści
%

\begin{multicols}{2}
	\footnotesize
	\tableofcontents
\end{multicols}

\noindent\makebox[\linewidth]{\rule{0.6\paperwidth}{0.4pt}}

% \begin{flushleft}
% 	\emph{Motto:}\\
% 	\textit{Pisanie raportu przywilejem każdego studenta.}
% \end{flushleft}


% { \color{red}\bf 
% \section*{Informacje o projekcie}


% W przypadku realizacji projektu przez dwie osoby (lub więcej osób), każdy z wykonawców projektu oddaje własną kopię sprawozdania/raportu, gdzie w jednym w punktów wymagany jest szczegółowy opis zrealizowanych zadań.
% }

\section{Wprowadzenie} 
\label{sec:wprowadzenie}

Celem projektu jest stworzenie gry typu "Platformer" w przestrzeni 3D i z widokiem z pierwszej
osoby

% Opis sekcji. Wzór jeśli trzeba bez numeracji
% \begin{displaymath}
% C^{-1}(C(m))=m \hspace{1cm} oraz \hspace{1cm}  |C(m)|<|m| .
% \end{displaymath}

% Wzór jeśli trzeba z numeracją:
% \begin{equation}
% L_{ave}=P(m_{1})L(m_{1})+...+P(m_{n})L(m_{n}),
% \end{equation}
% gdzie $L(m_{i})=-\lg(P(m_{i}))$ natomiast $P(m_{i})$ oznacza prawdopodobieństwo wystąpienia $m_{i}$ symbolu. 


% \begin{algorithm}
% \caption{-- Algorytm tworzenia kodu Shannona-Fano} \label{SF_code_making}
% \begin{algorithmic}
% \STATE{uporządkować zbiór symboli S malejąco ze względu na prawdopodobieństwo ich występowania}
% \STATE{$ShannonFano(S)$} \IF{$|S|=2$} \STATE{dopisz zero do kodu jednego elementu oraz jedynkę do drugiego kodu}
% \ELSIF{$|S|>1$} \STATE{ podziel S na dwa podciągi $S_{1}$ i $S_{2}$ tak aby różnica prawdopodobieństw tych
% podciągów była jak najmniejsza} \STATE{do każdego symbolu w $S_{1}$ dopisz zero a do każdego symbolu w $S_{2}$
% dopisz jeden} \STATE{$SannonFano(S_{1})$} \STATE{$SannonFano(S_{2})$} \ENDIF
% \end{algorithmic}
% \end{algorithm}



\section{Opis wkładu własnego w realizację projektu}

\section{Spis zastosowanych zasobów}

\section{Podsumowanie}


\end{document}